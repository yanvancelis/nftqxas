\chapter{Introdução}
\label{cap:introducao}

%Para começar a usar este \textit{template}, na plataforma \textit{ShareLatex}, vá nas opções (três barras vermelhas horizontais) no canto esquerdo superior da tela e clique em "Copiar Projeto" e dê um novo nome para o projeto. 

O mercado de artes digitais sempre se mostrou um lugar difícil para os artistas. Em um mercado onde a garantia de valores como legitimidade e unicidade é fundamental, a demasiada facilidade de reprodução fornecida por ferramentas digitais torna a garantia desses valores uma tarefa ingrata. O colecionismo, atividade comum a quem costuma adquirir obras de arte, é praticamente inexistente em contextos digitais. A unicidade gera uma exclusividade que é muito particular a esse segmento, diferenciando-o de outras obras criativas, como a música, cujo mercado encontrou seu caminho através dos streamings. No entanto, a música sempre teve grande apelo reprodutivo, através das rádios e mídias fabricadas em larga escala, logo, a capacidade de reprodução fornecida por meios digitais não representava uma barreira como representa para obras de arte. 

Essas dificuldades fazem com que boa parte dos artistas opte por não comercializar suas obras por meios digitais, dando preferência a impressos. Ainda assim, o risco da reprodução não consentida persiste, dada a popularidade de meios de cópia disponíveis. Nesse sentido, as NFTs surgem como uma interessante alternativa. Sigla para Non-Fungible Tokens, as NFTs pertencem ao contexto das Blockchains, a base técnica por trás do funcionamento de criptomoedas como Bitcoin e Ethereum. Uma NFT nada mais é do que um Token cuja natureza é única, graças as suas propriedades que permitem sua distinção de um Token para outro. Eles podem representar qualquer tipo de item, desde um ingresso de um jogo de futebol a uma obra de arte, e graças as características das Blockchain, que incluem descentralização e protocolos de consenso, tornaram-se uma opção importante para publicação e venda de artefatos como obras de arte. 






Se no mercado de artes tradicionais é possível garantir essa 





Tokens não fungíveis. Essa é a tradução livre para Non-Fungible Tokens, palavras que juntas formam a sigla NFT. Ela ficou famosa nos últimos anos com o advento de seu uso, ligado ao contexto do mercado das artes digitais e tendo sido inclusive um dos assuntos mais buscados por brasileiros no Google em 2022 . Sua existência está ligada as Blockchains, tecnologia base para o funcionamento de criptomoedas como Bitcoin e Ethereum. Um Token não fungível é um token cuja natureza é única, ou seja, é possível diferencia-lo de um para outro graças as suas propriedades, diferente de Tokens convencionais. 

%Testando o símbolo $\symE$

%\lipsum[5]  % Simulador de texto, ou seja, é um gerador de lero-lero.

%	\begin{alineas}
%		\item Lorem ipsum dolor sit amet, consectetur adipiscing elit. Nunc dictum sed tortor nec viverra.
%		\item Praesent vitae nulla varius, pulvinar quam at, dapibus nisi. Aenean in commodo tellus. Mauris molestie est sed justo malesuada, quis feugiat tellus venenatis.
%		\item Praesent quis erat eleifend, lacinia turpis in, tristique tellus. Nunc dictum sed tortor nec viverra.
%		\item Mauris facilisis odio eu ornare tempor. Nunc dictum sed tortor nec viverra.
%		\item Curabitur convallis odio at eros consequat pretium.
%	\end{alineas}
	

	
