\chapter{Fundamentação teórica}
\label{cap:fundamentacao-teorica}

\section{Non-Fungible Tokens}
\label{sec:non-fungible tokens}

Non-Fungible \textit{token}s, ou \textit{tokens} não fungíveis em tradução livre, são as palavras que compõem a sigla NFT. Ela ganhou popularidade nos últimos anos, tendo sido eleita a palavra do ano 2020 pelo dicionário Collins \footnote{Disponível em: <https://g1.globo.com/tecnologia/noticia/2021/11/24/nft-e-eleita-a-palavra-do-ano-2020-pelo-dicionario-collins.ghtml> Acesso em: 12 jan, 2023}, continuando popular em 2022, sendo destaque nas pesquisas realizadas por brasileiros no Google \footnote{Disponível em: <https://einvestidor.estadao.com.br/ultimas/nft-destaque-buscas-google-2022> Acesso em: 12 jan, 2023}. Essa popularidade se deve as movimentações de grandes cifras monetárias em torno de um novo tipo de item virtual que inundaram os noticiários e as redes sociais, especialmente com o envolvimento de pessoas famosas, a exemplo do atleta Neymar Jr, que na ocasião investiu cerca de 790 mil reais em um desses itens \cite{Andrade}. Esses itens virtuais nada mais eram que imagens, desenhos e fotografias, entre outros tipos de artefatos visuais, comercializados a partir de redes \textit{\textit{blockchain}}.

\subsection{Blockchain}
\label{subsec: blockchains e criptomoedas}

\textit{Blockchain}, ou cadeia de blocos em tradução livre, é uma tecnologia de armazenamento de dados distribuída que se baseia no uso de um livro-razão. Esse livro, também conhecido como \textit{ledger}, é um arquivo que mantém uma lista incremental de registos de transações, em blocos ligados criptograficamente, protegidos de adulteração e revisão \cite{Voshmgir, Lyra}. Esse arquivo é mantido de forma distribuída, sendo público e auditável por todos os participantes da rede, que possuem uma cópia completa e atualizada conforme novos blocos são inseridos. 

Dessa forma, todos podem visualizar e auditar todas as transações já realizadas. Para cada nova inserção, os atores da rede precisam validar a autenticidade da transação, através do protocolo de consenso. A \textit{blockchain} está por trás do fenômeno das criptomoedas, ganhando poularidade a partir do surgimento do Bictoin em 2009, ganhando novas aplicações também em outros domínios nos últimos anos.

\subsection{Tokens}
\label{subsec: tokens}
Em uma \textit{blockchain}, \textit{token}s são unidades de troca e podem assumir diversas representações, incluindo moedas, registros, identidades, entre outros \cite{Antonopoulos, Voshmgir}. Um \textit{token} não fungível é um tipo de \textit{token} cuja natureza é única, com propriedades variáveis capazes de diferenciar uns dos outros \cite{Voshmgir}. Segundo o Dicionário Priberam da Língua Portuguesa, o adjetivo "fungível" significa algo que se gasta após primeiro o uso, ou seja, que é descartável ou substituível, assim, a não-fungibilidade se refere a unicidade representada por esse tipo de \textit{token}, a sua exclusividade. Por serem propagados em redes \textit{blockchain}, os NFT trazem consigo as garantias desse tipo de rede, que consistem em mecanismos públicos e distribuídos de validação, propagação e comercialização desses ativos. Esse conjunto de fatores fez com que os NFT se tornassem um interessante meio para propagação de obras artísticas, embora não fiquem limitados a elas, conforme explica Voshmgir:

\begin{citacao}
	 Os NFT podem também representar identidades e certificados, tais como licenças, graus, chaves, passes, identidades, testamentos, direitos de voto, bilhetes, \textit{token}s de fidelização, direitos de autor, garantias, licenças de software, dados médicos, e certificados de qualquer tipo, tais como cadeias de fornecimento ou certificados de arte \cite{Voshmgir}
\end{citacao}

No entanto, é interessante notar como as possibilidades oferecidas pelos NFT resolvem em grande parte as dores dos artistas e outros profissionais que fornecem produtos cuja exclusividade é posta em cheque pela natureza intuitiva das mídias digitais. Conforme explica o fotógrafo Alex Montesso, "Um NFT é um certificado de propriedade de um ativo digital que não pode ser alterado ou falsificado", concluindo que "essa certificação garante a rastreabilidade e certifica a peça como autêntica" \cite{Montesso}. Para o escritor Logan Kugler, essa capacidade (de certificar uma propriedade digital) exclusiva era algo impensável até o surgimento dos NFT: 

\begin{citacao}
Essa dinâmica cria uma simples, mas poderosa forma de como trabalhar com artes digitais, tornando-as exclusivas. Uma vez cunhado na \textit{blockchain} Ethereum, o NFT é representado em um livro-razão público que não pode ser alterado. Ao possuir o \textit{token}, você prova ser dono da obra de arte. Não há nada que impeça sua visualização online, ou mesmo sua cópia e compartilhamento, mas sem a NFT, não é possível fingir a posse da obra de arte (...) \cite{Kugler}
\end{citacao}

Cunhar é o verbo que se refere a publicação de um NFT, ou seja, a sua inserção no livro-razão pertencente a uma \textit{blockchain}. É a essa estrutura que se deve a capacidade de um NFT em resguardar a exclusividade de uma obra, bem como prover garantias aos seus donos, sejam criadores ou colecionadores. Isso também garante a capacidade de transferir obras para outros indivíduos, permitindo a revenda desses itens.

Outra característica importante introduzida pelos NFT foi a capacidade de gerar escassez artificial em obras digitais de forma escalável e eficiente \cite{Kugler}. Um arquivo digital pode ser copiado infinitamente sem que perca suas características, algo que pode inviabilizar o controle de reprodução e, por consequência, a capacidade de garantir a autenticidade de uma obra. Para Guilherme Preger, os NFT vão num sentido contrário à da reprodutibilidade dos meios digitais, valorizando a autenticidade e a singularidade dessas obras \cite{Preger}. Ao cunhar um NFT, o artista pode definir uma quantidade limitada de itens referentes a uma determinada obra, garantindo que somente quem tiver posse desses \textit{token}s será um proprietário verdadeiro dessas obras. 

Transferir qualquer tipo de titularidade (ou posse) de artefatos digitais também é uma demanda de vários segmentos. Ao longo dos últimos anos, várias tentativas de controlar a distribuição de produtos como e-books e músicas por meios digitais sempre esbarraram em limitações de ordem técnica e ou mesmo de direitos básicos do consumidor, como revenda e até mesmo de empréstimo. O DRM\footnote{Mecanismo complexo de proteção baseado em inúmeras tecnologias com objetivo de vincular conteúdo específico a um determinado grupo de permissões de acesso e uso, em operação integrada a instrumentos de monitoramento e registro de consumo \cite{Vieira}}, a mais popular dessas tentativas, além de limitar o direito dos consumidores, privilegia plataformas (limitando a experiência e liberdade de uso) e sequer conseguem garantir os direitos dos autores, dadas as vulnerabilidades nesse sistema. Os NFT permitem operações de troca, revenda e até a possibilidade de presentear com esses itens digitais, desde que ambas as partes estejam em uma mesma \textit{blockchain}, o que não representa um grande problema, já que carteiras digitais \footnote{explicar o que é uma carteira digital} mais populares permitem a conexão com várias redes distintas. 

\subsection{Outras aplicações}
\label{subsec: outras aplicações}

Originalmente, NFT estão disponíveis através da plataforma Ethereum, uma \textit{blockchain} de código aberto que executa programas chamados contratos inteligentes e permite que desenvolvedores criem variadas aplicações descentralizadas \cite{Antonopoulos}. Entre essas aplicações, estão outras criptomoedas e até o objeto deste trabalho, os NFT. Em uma \textit{blockchain}, um contrato inteligente, ou \textit{smart-contracts} é uma ferramenta de gestão de direitos que pode formalizar e executar acordos auto-executáveis entre participantes não confiáveis \cite{Voshmgir}. Por meio desses contratos, é possível embutir em um NFT um conjunto de regras nos termos de venda, que podem incluir inclusive o pagamento automático de royalties ao artista sempre que a obra mudar de mãos em uma ocasião de revenda, por exemplo \cite{Kugler}.



\section{Artes digitais}
\label{sec:non-fungible tokens}

\subsection{Formas de arte digital}
\label{subsec: formas de arte digital}


\subsection{Ferramentas utilizadas}
\label{subsec: ferramentas utilizadas}


\subsection{Criptoarte}
\label{subsec: criptoarte}




\section{Mercado da arte}
\label{sec:non-fungible tokens}

\subsection{Galerias, museus e outros}
\label{subsec: galerias, museus e outros}

\subsection{Artistas e consumidores}
\label{subsec: artistas e consumidores}

\subsection{O valor da obra de arte}
\label{subsec: o valor das obras}

\subsection{O mercado de artes digitais}
\label{subsec: o mercado digital}















