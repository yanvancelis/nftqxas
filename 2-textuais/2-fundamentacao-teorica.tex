\chapter{Fundamentação teórica}
\label{cap:fundamentacao-teorica}

\section{Non-Fungible Tokens}
\label{cap:non-fungible tokens}

Non-Fungible Tokens, ou Tokens não Fungíveis em tradução livre, são as palavras que compõem a sigla NFT. Ela ganhou popularidade nos últimos anos, tendo sido eleita a palavra do ano 2021 pelo dicionário Collins \footnote{Disponível em: <https://g1.globo.com/tecnologia/noticia/2021/11/24/nft-e-eleita-a-palavra-do-ano-2020-pelo-dicionario-collins.ghtml> Acesso em: 12 jan, 2023}, continuando popular em 2022, sendo destaque nas pesquisas realizadas por brasileiros no Google \footnote{Disponível em: <https://einvestidor.estadao.com.br/ultimas/nft-destaque-buscas-google-2022> Acesso em: 12 jan, 2023}. Essa popularidade se deve as movimentações milionárias em torno de um novo tipo de objeto virtual que inundaram os noticiários e as redes sociais, especialmente com o envolvimento de pessoas famosas e investimentos de grande ordem financeira\footnote{Disponível em: <https://einvestidor.estadao.com.br/criptomoedas/bored-ape-nft-neymar-valor> Acesso em: 14 jan, 2023} . Basicamente, esses objetos virtuais nada mais eram que imagens, desenhos e fotografias, entre outros tipos de artefatos visuais, comercializados a partir de redes Blockchain, a essa altura já consolidadas através das criptomoedas, um tipo de moeda digital que usa criptografia para aplicação de segurança e funcionam de forma descentralizada, sem autoridades centrais ou controle de terceiros \cite{Chaves}.

Um Token não fungível é um tipo de Token cuja natureza é única, com propriedades variáveis capazes de diferenciar uns dos outros \cite{Voshmgir}. Um Token é uma referência a um ativo propagado em uma Blockchain, embora seja um termo com variados usos. Fungível é algo que se desgasta após o uso, é descartável ou substituível, assim, a não-fungibilidade se refere a unicidade representada por esse tipo de Token, a sua exclusividade. Por serem propagados em redes Blockchain, as NFTs trazem consigo as garantias desse tipo de rede, que fornecem mecanismos públicos e distribuídos de validação, propagação e comercialização desses ativos. Esse conjunto de fatores fez com que os NFTs se tornassem um interessante meio para propagação de obras artísticas, embora não fique limitado a esse tipo de artefato, conforme explica Voshmgir:

\begin{citacao}
	 Os NFT podem também representar identidades e certificados, tais como licenças, graus, chaves, passes, identidades, testamentos, direitos de voto, bilhetes, tokens de fidelização, direitos de autor, garantias, licenças de software, dados médicos, e certificados de qualquer tipo, tais como cadeias de fornecimento ou certificados de arte (2021, p233)	
\end{citacao}

No entanto, é interessante notar como as características oferecidas pelos NFTs resolvem em grande parte as dores dos artistas e outros profissionais que fornecem produtos cuja originalidade é posta em cheque pela natureza intuitiva das mídias digitais. Um NFT é um certificado de propriedade de um ativo digital que não pode ser alterado ou falsificado (...), essa certificação garante a rastreabilidade e certifica a peça como autêntica (Montesso, 2022, p20). Para Kugler (2021, p19), a capacidade de certificar uma propriedade exclusiva era algo impensável até o surgimento dos NFTs, especialmente se pensarmos no contexto digital: 

\begin{citacao}
Essa dinâmica cria uma simples, mas poderosa forma de como trabalhar com artes digitais, tornando-as exclusivas. Uma vez cunhado na blockchain Ethereum, o NFT é representado em um livro-razão público que não pode ser alterado. Ao possuir o token, você prova ser dono da obra de arte. Não há nada que impeça sua visualização online, ou mesmo sua cópia e compartilhamento, mas sem a NFT, não é possível fingir a posse da obra de arte (...)(Kugler, 2021, p19)	
\end{citacao}

Cunhar é o verbo que se refere a publicação de um NFT, ou seja, a sua inserção no livro-razão pertencente a uma blockchain, como o Ethereum \footnote{Conceito de Ethereum aqui}. Um livro-razão, também conhecido como Ledger, é um arquivo que mantém uma lista incremental de registos de transações, em blocos “acorrentados” criptograficamente e protegidos de adulteração e revisão (Voshmgir, 2021, p38). Em uma blockchain, esse arquivo é mantido de forma distribuída, sendo público e auditável por todos os participantes da rede. É a essa estrutura que se deve a capacidade de um NFT em resguardar a exclusividade de uma obra, bem como prover garantias aos seus donos, sejam criadores ou colecionadores. Isso também garante a capacidade de transferir obras para outros indivíduos, garantindo assim a possibilidade de revenda desses itens.

Outra característica importante introduzida pelos NFTs foi a capacidade de gerar escassez artificial em obras digitais de forma escalável e eficiente (Kugler, 2021, p19). Um arquivo digital possui possibilidades infinitas de cópias sem que perca qualquer tipo de característica original, o que inviabiliza o simples compartilhamento de uma obra, ainda que seja remunerada, pois não há nenhum tipo de controle de cópias, inviabilizando o direito de autoria e a garantia de propriedade do comprador. Ao cunhar uma NFT, o artista pode definir uma quantidade limitada de itens referentes a uma determinada obra, garantindo que somente quem tiver posse desses tokens será um proprietário verdadeiro dessas obras. 

Transferir qualquer tipo de titularidade (ou posse) de artefatos digitais sempre foi uma tarefa complicada, para não dizer impossível. Ao longo dos últimos anos, várias tentativas de controlar a distribuição de artefatos como e-books e músicas por meios digitais sempre esbarraram em limitações básicas de liberdade para quem consome esses produtos, incluindo imitações de revenda e até mesmo de empréstimo. O DRM\footnote{conceito de DRM aqui}, o exemplo mais famoso dentre essas tentativas, se mostra ainda hoje uma tecnologia que inviabiliza a real posse, já que impede até mesmo o acesso desses bens por plataformas que não sejam a da parte vendedora, não sendo possível, por exemplo, acessar um e-book adquirido na Amazon em uma ferramenta apartada como o Calibre.

Falar de transferências de NFT aqui e conectar com contratos inteligentes e suas regras

NFTs utilizam contratos inteligentes para resolver esse problema. Em uma blockchain, um contrato inteligente é uma ferramenta de gestão de direitos que pode formalizar e executar acordos auto-executáveis entre participantes não confiáveis (Voshmgir, 2021, p137). Por meio desses contratos, é possível embutir em um NFT um conjunto de regras que visam garantir os direitos 













