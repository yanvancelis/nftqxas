\chapter{Fundamentação teórica}
\label{cap:fundamentacao-teorica}

\section{Non-Fungible Tokens}
\label{sec:non-fungible tokens}

Non-Fungible \textit{token}s, ou \textit{tokens} não fungíveis em tradução livre, são as palavras que compõem a sigla NFT. Ela ganhou popularidade nos últimos anos, tendo sido eleita a palavra do ano 2020 pelo dicionário Collins \footnote{Disponível em: <https://g1.globo.com/tecnologia/noticia/2021/11/24/nft-e-eleita-a-palavra-do-ano-2020-pelo-dicionario-collins.ghtml> Acesso em: 12 jan, 2023}, continuando popular em 2022, sendo destaque nas pesquisas realizadas por brasileiros no Google \footnote{Disponível em: <https://einvestidor.estadao.com.br/ultimas/nft-destaque-buscas-google-2022> Acesso em: 12 jan, 2023}. Essa popularidade se deve as movimentações de grandes cifras monetárias em torno de um novo tipo de item virtual que inundaram os noticiários e as redes sociais, especialmente com o envolvimento de pessoas famosas, a exemplo do atleta Neymar Jr, que na ocasião investiu cerca de 790 mil reais em um desses itens \cite{Andrade}. Esses itens virtuais nada mais eram que imagens, desenhos e fotografias, entre outros tipos de artefatos visuais, comercializados a partir de redes \textit{\textit{blockchain}}.

\subsection{Blockchain}
\label{subsec: blockchains e criptomoedas}

\textit{Blockchain}, ou cadeia de blocos em tradução livre, é uma tecnologia de armazenamento de dados distribuída que se baseia no uso de um livro-razão. Esse livro, também conhecido como \textit{ledger}, é um arquivo que mantém uma lista incremental de registos de transações, em blocos ligados criptograficamente, protegidos de adulteração e revisão \cite{Voshmgir, Lyra}. Esse arquivo é mantido de forma distribuída, sendo público e auditável por todos os participantes da rede, que possuem uma cópia completa e atualizada conforme novos blocos são inseridos. 

Dessa forma, todos podem visualizar e auditar todas as transações já realizadas. Para cada nova inserção, os atores da rede precisam validar a autenticidade da transação, através do protocolo de consenso. A \textit{blockchain} está por trás do fenômeno das criptomoedas, ganhando poularidade a partir do surgimento do Bictoin em 2009, ganhando novas aplicações também em outros domínios nos últimos anos.

\subsection{Tokens}
\label{subsec: tokens}
Em uma \textit{blockchain}, \textit{token}s são unidades de troca e podem assumir diversas representações, incluindo moedas, registros, identidades, entre outros \cite{Antonopoulos, Voshmgir}. Um \textit{token} não fungível é um tipo de \textit{token} cuja natureza é única, com propriedades variáveis capazes de diferenciar uns dos outros \cite{Voshmgir}. Segundo o Dicionário Priberam da Língua Portuguesa, o adjetivo "fungível" significa algo que se gasta após primeiro o uso, ou seja, que é descartável ou substituível, assim, a não-fungibilidade se refere a unicidade representada por esse tipo de \textit{token}, a sua exclusividade. Por serem propagados em redes \textit{blockchain}, os NFT trazem consigo as garantias desse tipo de rede, que consistem em mecanismos públicos e distribuídos de validação, propagação e comercialização desses ativos. Esse conjunto de fatores fez com que os NFT se tornassem um interessante meio para propagação de obras artísticas, embora não fiquem limitados a elas, conforme explica Voshmgir:

\begin{citacao}
	 Os NFT podem também representar identidades e certificados, tais como licenças, graus, chaves, passes, identidades, testamentos, direitos de voto, bilhetes, \textit{token}s de fidelização, direitos de autor, garantias, licenças de software, dados médicos, e certificados de qualquer tipo, tais como cadeias de fornecimento ou certificados de arte \cite{Voshmgir}
\end{citacao}

No entanto, é interessante notar como as possibilidades oferecidas pelos NFT resolvem em grande parte as dores dos artistas e outros profissionais que fornecem produtos cuja exclusividade é posta em cheque pela natureza intuitiva das mídias digitais. Conforme explica o fotógrafo Alex Montesso, "Um NFT é um certificado de propriedade de um ativo digital que não pode ser alterado ou falsificado", concluindo que "essa certificação garante a rastreabilidade e certifica a peça como autêntica" \cite{Montesso}. Para o escritor Logan Kugler, essa capacidade (de certificar uma propriedade digital exclusiva) era algo impensável até o surgimento dos NFT: 

\begin{citacao}
Essa dinâmica cria uma simples, mas poderosa forma de como trabalhar com artes digitais, tornando-as exclusivas. Uma vez cunhado na \textit{blockchain} Ethereum, o NFT é representado em um livro-razão público que não pode ser alterado. Ao possuir o \textit{token}, você prova ser dono da obra de arte. Não há nada que impeça sua visualização online, ou mesmo sua cópia e compartilhamento, mas sem a NFT, não é possível fingir a posse da obra de arte (...) \cite{Kugler}
\end{citacao}

Cunhar é o verbo que se refere a publicação de um NFT, ou seja, a sua inserção no livro-razão pertencente a uma \textit{blockchain}. É a essa estrutura que se deve a capacidade de um NFT em resguardar a exclusividade de uma obra, bem como prover garantias aos seus donos, sejam criadores ou colecionadores. Isso também garante a capacidade de transferir obras para outros indivíduos, permitindo a revenda desses itens.

Outra característica importante introduzida pelos NFT foi a capacidade de gerar escassez artificial em obras digitais de forma escalável e eficiente \cite{Kugler}. Um arquivo digital pode ser copiado infinitamente sem que perca suas características, algo que pode inviabilizar o controle de reprodução e, por consequência, a capacidade de garantir a autenticidade de uma obra. Para Guilherme Preger, os NFT vão num sentido contrário à da reprodutibilidade dos meios digitais, valorizando a autenticidade e a singularidade dessas obras \cite{Preger}. Ao cunhar um NFT, o artista pode definir uma quantidade limitada de itens referentes a uma determinada obra, garantindo que somente quem tiver posse desses \textit{token}s será um proprietário verdadeiro dessas obras. 

Transferir qualquer tipo de titularidade (ou posse) de artefatos digitais também é uma demanda de vários segmentos. Ao longo dos últimos anos, várias tentativas de controlar a distribuição de produtos como e-books e músicas por meios digitais sempre esbarraram em limitações de ordem técnica e ou mesmo de direitos básicos do consumidor, como revenda e até mesmo de empréstimo. O DRM\footnote{Mecanismo complexo de proteção baseado em inúmeras tecnologias com objetivo de vincular conteúdo específico a um determinado grupo de permissões de acesso e uso, em operação integrada a instrumentos de monitoramento e registro de consumo \cite{Vieira}}, a mais popular dessas tentativas, além de limitar o direito dos consumidores, privilegia plataformas (limitando a experiência e liberdade de uso) e sequer conseguem garantir os direitos dos autores, dadas as vulnerabilidades nesse sistema. Os NFT permitem operações de troca, revenda e até a possibilidade de presentear com esses itens digitais, desde que ambas as partes estejam em uma mesma \textit{blockchain}, o que não representa um grande problema, já que carteiras digitais \footnote{explicar o que é uma carteira digital} mais populares permitem a conexão com várias redes distintas. 

Originalmente, NFT estão disponíveis através da plataforma Ethereum, uma \textit{blockchain} de código aberto que executa programas chamados contratos inteligentes e permite que desenvolvedores criem variadas aplicações descentralizadas \cite{Antonopoulos}. Entre essas aplicações, estão outras criptomoedas e até o objeto deste trabalho, os NFT. Em uma \textit{blockchain}, um contrato inteligente, ou \textit{smart-contracts} é uma ferramenta de gestão de direitos que pode formalizar e executar acordos auto-executáveis entre participantes não confiáveis \cite{Voshmgir}. Por meio desses contratos, é possível embutir em um NFT um conjunto de regras nos termos de venda, que podem incluir inclusive o pagamento automático de royalties ao artista sempre que a obra mudar de mãos em uma ocasião de revenda, por exemplo \cite{Kugler}.

\subsection{Outras aplicações}
\label{subsec: outras aplicações}

Vimos em sessões anteriores que NFTs possuem diversas aplicações práticas, ainda que estejam muito ligadas ao uso com obras de artes. Um dos segmentos que mais se utilizaram dos NFTs para incrementarem as experiências de seus produtos foi o dos jogos. A característica mais básicas fornecidas por esses tokens permitiram que criadores de jogos criassem experiências cada vez mais personalizadas para jogadores. Os NFT podem ser utilizados para representar ativos dentro do jogo, permanecendo sob controle do jogador em vez do criador do jogo \cite{Voshmgir}. Um dos primeiros jogos a ganharem popularidade explorando o uso de NFTs foi o Cripto Kitties\footnote{referenciar o joguinho aqui}, um game que permite a criação, venda e compra de gatinhos virtuais. 

Outros segmentos também utilizam NFT buscando oferecer diferenciais, a exemplo do futebol. É possível citar o exemplo de clubes que aderem ao uso dos NFT como forma de aumentar a experiência dos torcedores, assim como obter fundos para os clubes. No Brasil, o Vasco da Gama é um dos pioneiros no uso de NFTs. Ainda em 2021, o club anunciou a sua primeira coleção de NFTs em referência a "Resposta Histórica\footnote{Publicada em 1924, a "resposta histórica" foi elaborada por José Augusto Prestes, então presidente do Vasco. Naquela época, o esporte bretão ainda bastante elitista e não permitia a participação de negros e pessoas de origem humilde, como operários \cite{Resposta}}", momento importante da história do club. Desde então, o club vem investindo nesse tipo de ativo, incluindo o lançamento de NFTs de camisas\footnote{Disponível em <https://www.lance.com.br/lancebiz/vasco-lanca-nfts-da-nova-colecao-de-uniformes.html>} e até ingressos de jogos \footnote{Disponível em <https://olhardigital.com.br/2022/01/08/pro/vasco-sera-o-primeiro-clube-brasileiro-a-vender-ingressos-com-nfts/>}. 



\section{Arte digital}
\label{sec:arte digital}
Arte digital é uma disciplina que agrupa manifestações artísticas realizadas por computador e elaboradas de forma digital \cite{Lieser}. Essa disciplina também pode ser referida por outros termos, incluindo arte de novas mídias, arte computacional, arte tecnológica, ciberarte, tecnoarte, arte eletrônica, arte informática, entre outros, não havendo assim um consenso em um único termo \cite{Gobira, Arantes}.  Em 2009, o Ministério da Cultura (MinC) definiu que Arte Digital "compreende a produção artística envolvendo arte, tecnologia e ciência em diálogo com outras áreas como Ciência da Computação, Robótica, Mecatrônica, Genética, Comunicação ... \cite{MINC}. Neste trabalho, optamos pelo uso da terminologia "arte digital", e traremos outros termos como uma espécie de sub-categoria conforme necessário. É interessante notarmos já na definição de Lieser (2010) uma diferença entre artes promovidas por meio dos computadores e aquelas elaboradas através deles, ou seja, cuja natureza da criação envolve obrigatoriamente o uso desses dispositivos, e aquelas apenas promovidas através deles, conforme ele exemplifica:

\begin{citacao}
Uma foto digitalizada não pode ser considerada arte digital, mesmo sendo muito boa. Mas uma imagem que tenha sido obtida por uma câmera da internet em Nova Iorque e segundos mais tarde seja vista em Berlim, essa sim, pode ser considerada arte digital. \cite{Lieser}	
\end{citacao}


Sendo assim, diz-se que é uma arte digital uma obra confeccionada tendo um computador como parte primordial do processo. Cabe nos atentarmos ao uso do termo 'computador', que talvez não reflita o atual momento, dada a variedade de dispositivos que utilizam um microprocessador, que embora sejam computadores em sua essência, não são necessariamente categorizados assim, a exemplo dos smartphones, tablets e consoles de video-game. Por isso, adotaremos o uso do termo 'aparelhos digitais', mais aberto e abrangente, especificando quando necessário.


\subsection{Tipos de arte digital}
\label{subsec: formas de arte digital}

As artes digitais podem se manifestar de diversas maneiras, começando  pelos formatos tradicionais, como as artes visuais, a fotografia, o cinema e a música. As artes visuais, quando obtidas através de aparelhos digitais e softwares aplicativos específicos, assim como as fotografias capturadas por aparelhos fotográficos digitais, que tendem a passar por manipulações mínimas em ilhas digitais\footnote{Falar o que é uma ilha digital}, podem ser consideradas artes digitais. 

Obras cinematográficas como Toy Story (1995) e a produção brasileira Cassiopeia (1996), os primeiros filmes a serem realizados quase que totalmente a partir de softwares de computação gráfica, e ainda obras mais recentes, como Vingadores e Star Wars, que popularizaram o uso de computação nas grandes produções do cinema, podem ser categorizadas como artes digitais. Embora efeitos visuais no cinema possam ser vistos desde muito antes, em filmes como o longa Viagem à Lua (Georges Melies) e a própria trilogia original de Star Wars (George Lucas), a evolução e a expansão do uso da computação para obter efeitos visuais cada vez mais sofisticados fez dessa prática um padrão na indústria. É possível que filmes inteiros sejam gravados por meio de captação de movimentos e chromakey\footnote{Explicar aqui sobre as telas verdes} e que todo o trabalho fique por conta de profissionais da pós-produção. 

No segmento musical, 'Bum bum tam tam", de MC Fioti, é um exemplo de produção artística possibilitada por um conjunto de aparelhos digitais. A canção, que se baseia em um sample\footnote{Definir sample} de 'Partita em lá menor', do alemão Johann Sebastian Bach, é uma produção caseira que se apoia inteiramente em aparelhos digitais para existir, começando pelo uso da internet para busca de referências. "Comecei a pesquisar alguns tipos de flauta, coisas antigas. E nisso eu achei a 'flautinha do Sebastian Bach", explica Fioti, que completa, "eu baixei e fiz o primeiro sample (...)". Na falta de um microfone, a voz foi gravada através do smartphone. Sem estrutura para produção, Fioti conta que dominou o programa de edição a partir de tutoriais no Youtube e realizou a edição em um notebook, nas suas palavras, "cheio de vírus, do pior que tem" \cite{Ortega}. 

Para além do uso de aparelhos digitais durante a produção, é possível perceber na fala de Fioti, em verbos como "baixar" e "samplear", ou seja, o ato de criar a partir de algo existente, uma linguagem muito característica das novas mídias, conforme descrição de Lev Manovich: a facilidade de desmontar e reorganizar objetos já existentes, bem como modifica-los e combina-los para criação de algo novo \cite{Martino}. Essas características, quando incorporadas as produções artísticas, refletem nessas obras uma espécie de hibridização. "Bum bum tam tam", uma música de Funk, gênero musical cuja as famosas batidas incorporam samples dos alemães do Kraftwerk, combinada a uma composição de música erudita do também alemão Bach, recombinadas, são capazes de originar um produto final totalmente novo.

O uso de aparelhos não adequados para realizar uma produção profissional também indica uma espécie de subversão desses aparelhos, já que o smartphone e o notebook "barato" não foram pensados para serem utilizados em uma produção dessa natureza, que geralmente requer um estúdio apropriado, muitas vezes inacessível para grande parte das pessoas. Conforme explica Arlindo Machado, o verdadeiro criador não se submete as determinações impostas pela indústria a lógica funcional desses aparelhos, manejando-os no sentido contrário ao que foram programados \cite{Machado}, ou seja, a produção artística digital não diz respeito apenas ao uso desses aparelhos, mas também sua superação dos limites impostos por estes. 

Saindo das manifestações tradicionais, as possibilidades trazidas junto a utilização de aparelhos digitais deram origem a outros tipos de manifestações, incluindo o surgimento da realidade virtual e aumentada, a arte generativa, animação digital, entre outras. Muitas obras artísticas são concretizadas utilizando um conjunto dessas manifestações, a exemplo de um jogo digital. Muitos jogos digitais são capazes de unir diferentes tipos de manifestação, como artes visuais, animação e música, unido a roteiros e até atuações feitas para interação também através de aparelhos digitais. 

\subsection{Criptoarte: NFT como suporte para artes digitais}
\label{subsec: criptoarte}

Em tópicos anteriores, explicamos como os NFT se tornaram uma interessante alternativa para difusão de artes digitais. A capacidade de garantir a exclusividade, escassez, autoria e até a posse de itens digitais a tornaram uma ferramenta formidável para artistas e colecionadores. Isso tornou a tecnologia quase um sinônimo de arte digital, embora  os NFT possam representar praticamente qualquer tipo de item. No entanto, essa associação que ajudou a popularizar os NFT e o que é chamado de criptoarte. Esse termo se refere a "forma de comércio baseada na criação, troca e venda de registros digitais que podem representar virtualmente quaisquer tipos de conteúdo de mídia publicados na internet" \cite{Menotti}. 

Assim, podemos dizer que os NFT servem como uma espécie de suporte para obras e outras produções artísticas digitais, que podem ser comercializadas através do mercado de criptoarte. Como sendo um tipo de mercado, há um conjunto de regras que ditam uma certa lógica de operação que não necessariamente está ligada ao ordenamento do mercado de artes tradicional. Como dito, os NFT herdam características das blockchains, que além de proverem as interessantes características citadas acima, também demandam um entendimento específico das regras e do funcionamento desse tipo de rede. Na próxima seção, trataremos desse tema e traçaremos um paralelo entre o mercado de criptoarte e o mercado de artes tradicional.


\section{Mercado da arte}
\label{sec:mercado de arte}

\subsection{Galerias, museus e outros}
\label{subsec: galerias, museus e outros}

\subsection{Artistas e consumidores}
\label{subsec: artistas e consumidores}

\subsection{O valor da obra de arte}
\label{subsec: o valor das obras}

\subsection{O mercado de artes digitais}
\label{subsec: o mercado digital}















